% !TEX program = xelatex
\documentclass{resume}
%\usepackage{zh_CN-Adobefonts_external} % Simplified Chinese Support using external fonts (./fonts/zh_CN-Adobe/)
%\usepackage{zh_CN-Adobefonts_internal} % Simplified Chinese Support using system fonts

\begin{document}
\pagenumbering{gobble} % suppress displaying page number

\name{Nhut-Nam Le}

\basicInfo{
  \email{18120061@student.hcmus.edu.vn} \textperiodcentered\ 
  \phone{(+84) 964 614 552} \textperiodcentered\ 
  \linkedin[lenhutnam298]{https://www.linkedin.com/in/lenhutnam298}
}
\section{Education}
\datedsubsection{\textbf{Ho Chi Minh University of Science (HCMUS)}, Ho Chi Minh City, Vietnam}{08.2020 -- Present}
\textit{BS-MS Program in Computer Science}\newline
\textit{Major} in Computer Science (Graduated) - Anticipated Graduation Date: 2024\newline
\datedsubsection{\textbf{Ho Chi Minh University of Science (HCMUS)}, Ho Chi Minh City, Vietnam}{08.2018 -- Present}
\textit{Bachelor's Science degree} in Information Technology (IT)\newline
\textit{Major} in Computer Science (Undergraduate) - Anticipated Graduation Date: 2022\newline
\textit{GPA: 7.95/10.0 (\href{https://drive.google.com/file/d/1U2fe7NfBfiRUw4pfSsZuHV7ICRix7x4_/view?usp=sharing}{Full Official Transcript})}
% \section{Experience}

% Reference Test
%\datedsubsection{\textbf{Paper Title\cite{zaharia2012resilient}}}{May. 2015}
%An xxx optimized for xxx\cite{verma2015large}
%\begin{itemize}
%  \item main contribution
%\end{itemize}
\section{Personal Projects}
\textbf{\href{https://github.com/NT-ThuHang/PPNCKH}{Applied Deep Learning for Breast Classification - Science Research Method}}
\begin{itemize}[parsep=0.5ex]
	\item Description: 
	\begin{itemize}
		\item In project, we will using Deep Learning, especially Transfer Learning with ResNet to classify breast cancer image (IDC Image)
		\item In final result, we got a good accuracy for this problem, about 86\% with our best model
		\item We also wrote a small paper in Vietnamese to report and presentation our results.
	\end{itemize}
	\item Using Python 3 (Jupyter Notebook)
\end{itemize}
\textbf{\href{https://github.com/nhutnamhcmus}{SincNet presentation}}
\begin{itemize}[parsep=0.5ex]
	\item Description: 
	\begin{itemize}
		\item In Introduction to Machine Learning Class, we present about the problem when processing speech signal, the idea and the architecture of SincNet
		\item In Recognition Class, we cover the traditional methods of voice biometrics, the identification features as well as some case study. Beside that, we also cover the state of the art of voice biometrics in recent years, with five related works in features extraction (d-vectors, j-vectors and x-vectors), in speaker classification by using Deep Learing such as multi-features domain and the SincNet Architecture.
	\end{itemize}
	\item Using Python 3 on Google Colab
\end{itemize}
\textbf{\href{https://github.com/nhutnamhcmus/ml-lab-02-classification}{Classification Project - Intro ML Course HCMUS}}
\begin{itemize}[parsep=0.5ex]
	\item Description: 
	\begin{itemize}
		\item Visualize and Analyze Plant Pathology data
		\item Data Preprocessing and Data Visualizaion with Python 3 (Matplotlib, Seaborn, Pandas)
		\item Experimental: Setup and training Neural Network to classify leaf diseases
		\item Testing and found very good results: With the Transfer Learning method with ResNet18, 34, 50 and EfficentNet B5 networks, the problem of disease leaf classification has been solved well with a relatively high accuracy after 5-fold of more than 97\%, with the technique of using dropout , K-Fold Cross Validation ensures minimizing, avoiding overfitting
	\end{itemize}
	\item Using Python 3 on Google Colab
\end{itemize}
\textbf{\href{https://github.com/nhutnamhcmus/lab-01-regression}{Regression Project - Intro ML Course HCMUS}}
\begin{itemize}[parsep=0.5ex]
	\item Description: 
	\begin{itemize}
		\item Visualize and Analyze Medical Insurance Cost data
		\item Data Preprocessing and Data Visualizaion with Python 3 (Matplotlib, Seaborn, Pandas)
		\item Analyze the correlation between features in Medical Insurance Cost data
		Experimental some Regression Algorithms with implemented in Python 3 such as Linear Regression, Ridge Regression, Lasso Regression, Random Forest Regressor and Polynomial Regression
	\end{itemize}
	\item Using Python 3 on Google Colab
\end{itemize}
\textbf{\href{https://github.com/nhutnamhcmus/nlp-word-addin}{Intro NLP Final Project}}
\begin{itemize}[parsep=0.5ex]
	\item Description: Build an simple Add-in for Microsoft Word to tokenization and statistical words.
	\item Programming language: JavaScripts
	\item Framework: Office Add-ins for MS Office
\end{itemize}
\textbf{\href{https://github.com/nhutnamhcmus/thpt-qg-2018}{Analysis for National High School Graduation Exam 2018}}
\begin{itemize}[parsep=0.5ex]
	\item Description: Collecting and visualization data to investigate information in data, find cheating in exam at Ha Giang in 2018.
	\item Programming language: Python 3/ Jupyter Notebook
\end{itemize}

\textbf{\href{https://github.com/nhutnamhcmus/wine-linear-discriminant-analysis}{Wine Dataset Classification using Linear Discriminant Analysis}}
\begin{itemize}[parsep=0.5ex]
	\item Description: Implementation Linear Discriminant Analysis on Wine Dataset.
	\item Programming language: Python 3/ Jupyter Notebook
\end{itemize}

\textbf{\href{https://github.com/nhutnamhcmus/Chat-App}{A small Java Chat Application}}
\begin{itemize}[parsep=0.5ex]
	\item Description: Desktop chat app using Java Swing, Networking, Multi-threading. This is Final Project for Topics in Java Application Course HCMUS. 
	\item Programming language: Java 
	\item Framework: Hibernate
\end{itemize}

\textbf{\href{https://github.com/nhutnamhcmus/Dictionary}{A small English-Vietnamese Dictionary Application}}
\begin{itemize}[parsep=0.5ex]
	\item Description: Project midtern of Topics in Java Application Course HCMUS, learning about some important topic Generic Programming, Collections \& Swing in Java with using Jetbrains Intellij.
	\item Programming language: Java 
\end{itemize}

\textbf{\href{https://github.com/nhutnamhcmus/simple-med}{A small web application visualize Minimum Edit Distance algorithm}}
\begin{itemize}[parsep=0.5ex]
	\item Description: Simple web application using HTML, CSS and JavaScripts to visualize Minimum Edit Distance algorithm.
	\item Programming language: JavaScripts 
\end{itemize}

\textbf{\href{https://github.com/nhutnamhcmus/dip}{Basic implement some traditional concepts in Digital Image Processing }}
\begin{itemize}[parsep=0.5ex]
	\item Description: Basic implement some traditional concepts in Digital Image Processing such as 1D, 2D Convolutional, Color Transformation, Geometry Transformation, Edge Detection using Sobel, Prewitt, Laplace window, Filter image in spatial domain.
	\item Programming language: Python 3 (using Jupyter Notebook), C++
\end{itemize}


\section{Expertise}
\textbf{Data Structure - Algorithms}
\begin{itemize}[parsep=0.5ex]
	\item Know and understand basic concepts of Data Structures
	\item Good understand about fundamentals of Algorithm
\end{itemize}
\textbf{Programming Paradigms}
\begin{itemize}[parsep=0.5ex]
	\item Understand basic concepts of Object - Oriented Programming.
	\item Know about SOLID Principle
	\item Know about some popular traditional Design Patterns such as Singleton, Proxy
\end{itemize}
\textbf{Machine Learning/ Computer Vision/ NLP}
\begin{itemize}[parsep=0.5ex]
	\item Familiar with OpenCV in Python 3 or C++.
	\item Ability to apply analytical and problem solving skills.
	\item Solid understanding of linear algebra, geometry, meshes, image processing.
	\item Familiar working with PyTorch
\end{itemize}
\textbf{Data Visualization}
\begin{itemize}
	\item Have a good knowledge about some algorithms for reduce and presentation data such as PCA, LDA
	\item Basic knowledge about data visualization libraries with Python 3 like Matplotlib, Seaborn, Plotly
\end{itemize}
\section{Skills}
\begin{itemize}[parsep=0.5ex]
	\item Programming Languages: familiar with using Python 3, C/C++ at intermediate level, know Java, JavaScript
	\item Platform: Linux (Ubuntu, Arch Linux), Windows.
	\item IDE - Text Editor \& Tooling: Jetbrains IDE (Intellij, Pycharm), VS Code, Git for Version Control.
\end{itemize}

% \datedline{\textit{\nth{1} Prize}, Award on xxx }{Jun. 2013}
% \datedline{Other awards}{2015}

\section{Miscellaneous}
\begin{itemize}[parsep=0.5ex]
	\item GitHub: https://github.com/nhutnamhcmus
	\item Languages: English - Intermediate, Vietnamese for native speaker
	\item Certificates
	\begin{itemize}
		\item \href{https://www.coursera.org/account/accomplishments/certificate/YSBWXVW4WRZL}{Mathematics for Machine Learning: Linear Algebra}
		\item \href{https://www.hackerrank.com/certificates/df26a981574f}{HackerRank Python (Basic) Certificate}
		\item \href{https://www.hackerrank.com/certificates/b2c2333af97c}{HackerRank Problem Solving (Basic) Certificate}
		\item \href{https://www.hackerrank.com/certificates/9802ec0ba5b1}{HackerRank Java (Intermediate) Certificate}
		\item \href{https://www.datacamp.com/statement-of-accomplishment/track/5d1fb70723961aac5c2c69fa76880a86e5a6979e}{Datacamp Python Programmer Certificate}
	\end{itemize}	
\end{itemize}

%% Reference
%\newpage
%\bibliographystyle{IEEETran}
%\bibliography{mycite}
\end{document}